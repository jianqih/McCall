\documentclass[11pt,a4paper]{article}

\usepackage{graphicx,hyperref,amsmath,natbib,bm,url}
\usepackage[utf8]{inputenc} % include umlaute

% -------- Define new color ------------------------------
\usepackage{color}
\definecolor{mygrey}{rgb}{0.85,0.85,0.85} % make grey for the table
%   }}
\definecolor{darkblue}{rgb}{0.055,0.094,0.7}
\definecolor{darkred}{rgb}{0.6,0,0}

% --------------------------------------------------------
\usepackage{microtype,todonotes}
\usepackage[australian]{babel} % change date to to European format
\usepackage[a4paper,text={14.5cm, 25.2cm},centering]{geometry} % Change page size
\setlength{\parskip}{1.2ex} % show new paragraphs with a space between lines
\setlength{\parindent}{0em} % get rid of indentation for new paragraph
\clubpenalty = 10000 % prevent "orphans" 
\widowpenalty = 10000 % prevent "widows"

\hypersetup{pdfpagemode=UseNone} % un-comment this if you don't want to show bookmarks open when opening the pdf

\usepackage{mathpazo} % change font to something close to Palatino

\usepackage{longtable, booktabs, tabularx} % for nice tables
\usepackage{caption,fixltx2e}  % for nice tables
\usepackage[flushleft]{threeparttable}  % for nice tables

\newcommand*{\myalign}[2]{\multicolumn{1}{#1}{#2}} % define new command so that we can change alignment by hand in rows
 
% -------- For use in the bibliography -------------------
\usepackage{multicol}
\usepackage{etoolbox}
\usepackage{relsize}
\setlength{\columnsep}{1cm} % change column separation for multi-columns
% \patchcmd{\thebibliography}
%   {\list}
%   {\begin{multicols}{2}\smaller\list}
%   {}
%   {}
% \appto{\endthebibliography}{\end{multicols}}
% % --------------------------------------------------------


\usepackage[onehalfspacing]{setspace} % Line spacing

\usepackage[marginal]{footmisc} % footnotes not indented
\setlength{\footnotemargin}{0.2cm} % set margin for footnotes (so that the number doesn't stick out) 

\usepackage{pdflscape} % for landscape figures

\usepackage[capposition=top]{floatrow} % For notes below figure

% -------- Use the following two lines for making lines grey in tables -------------------
\usepackage{color, colortbl}
\usepackage{multirow}
% ----------------------------------------------------------------------------------------
\newtheorem{defn}{Definition}[section]
\newtheorem{reg}{Rule}[section]
\newtheorem{exer}{Exercise}[section]
\newtheorem{note}{Note}[section]
\newtheorem{expl}{Example}[section]
\newtheorem{pro}{properties}[section]
\newtheorem{asm}{Assumption}
\newtheorem{them}{Theorem}[section]
\newtheorem{rmk}{Remark}[section]
\newtheorem{coro}{Corollary}[section]
\newtheorem{lem}{Lemma}[section]
\newtheorem{pros}{Proposition}[section]


\usepackage{dsfont}
% \tcbset{highlight math style={enhanced,
%     colframe=red!60!black,colback=yellow!50!white,arc=4pt,boxrule=1pt,
\hypersetup{colorlinks=true,           % put a box around links
  linkbordercolor = {1 0 0}, % the box will be red
  pdfborder = {1 0 0},       % 
  % bookmarks=true,            % PDF will contain an index on the RHS
  urlcolor=darkred,
  citecolor=darkblue,
  linkcolor=darkred
}
% The following defines a footnote without a marker
\newcommand\blfootnote[1]{%
  \begingroup
  \renewcommand\thefootnote{}\footnote{#1}%
  \addtocounter{footnote}{-1}%
  \endgroup
}

% -------- Customize page numbering ----------------------------
\usepackage{fancyhdr}
\usepackage{lastpage}
 \pagestyle{fancy}
\fancyhf{} % get rid of header and footer
 \cfoot{ \footnotesize{ \thepage } }

% \rhead{\small{\textit{Jianqi Huang}}}
% \lhead{\small{\textit{Heterogeneous Agents'n OLG model}}}
\renewcommand{\headrulewidth}{0pt} % turn off header line
% --------------------------------------------------------------

\begin{document}
\section{Model}\label{sec:model}
Motivation: what determines the level of employment and unemployment in the economy. 
Textbook: labor supply, labor demand, and unemployment as ``leisure'' .
Alternative: labor market frictions. 

Further question: how should labor market frictions be modeled?
Search and matching: costly process of workers finding the right jobs.

\paragraph{McCall Partial Equilibrium Search Model}
\begin{itemize}
    \item The simplest model of search frictions.
    \item Problem of an individual getting draws from a given wage distribution.
    \item Decision: which jobs to accept and which to reject. 
    \item Sequential search typically more reasonable.
\end{itemize}
\paragraph{Environment}
Risk neutral individual, discrete time. Linear preferences: $$ \sum_{t=0}^\infty \beta^t c_t $$
Start as unemployed, with consumption equal to $b$. Wage offer $w$ drawn from a distribution $F(w)$. If she accepts, she will be employed at that job forever. Net Present Value of accepting a wage offer $w$ is $w/(1-\beta)$ since $\frac{w}{1-\beta} = w + \beta w +\beta^2 w + \cdots$.

We could use probability to describe decision rule $$ a_t : \mathcal{W}\to [0,1] $$ where $\mathcal{W}$ is the set of wage offers.
Dynamic programming problem: $$ V(w) = \max \left\{ \frac{w}{1-\beta}, b + \beta \int V(w') \mathrm{d}F(w') \right\} $$ 

Reservation wage given by $$ \frac{R}{1-\beta} =  b+\int_{\mathcal{W}} V(w')\mathrm{d}F(w') $$
For all $w\geq R$, $V(w) = \frac{w}{1-\beta}$.
Therefore, $$ \int_{\mathcal{W}}V(w)\mathrm{d}F(w) = \frac{R F(R)}{1-\beta} +\int_{w\geq R}\frac{w}{1-\beta} \mathrm{d}F(w) $$
Combining with reservation wage equation, $$ \frac{R}{1-\beta} = b +\beta \left[\frac{R F(R)}{1-\beta} +\int_{w\geq R} \frac{w}{1-\beta}\mathrm{d}F(w)\right] $$
Rewriting, $$ \int_{w<R} \frac{R}{1-\beta} \mathrm{d}F(w) + \int_{w\geq R}\frac{R}{1-\beta}\mathrm{d}F(w) = b +\beta \left[\frac{R F(R)}{1-\beta} +\int_{w\geq R} \frac{w}{1-\beta}\mathrm{d}F(w)\right] $$
Subtracting $\beta R \int_{w\geq R} \mathrm{d}F(w)/(1-\beta)+\beta R \int_{w< R} \mathrm{d}F(w)/(1-\beta)$ from both sides,
$$ \begin{aligned}
    \int_{w<R} \frac{R}{1-\beta} \mathrm{d}F(w) + \int_{w\geq R}\frac{R}{1-\beta}\mathrm{d}F(w)-\beta R \int_{w\geq R} \mathrm{d}F(w)/(1-\beta)+\beta R \int_{w< R} \mathrm{d}F(w)/(1-\beta)\\ =  b+\beta\left[\int_{w\geq R}\frac{w-R}{1-\beta}d F(w)\right] 
\end{aligned}$$
Collecting terms, we obtain
\begin{equation}
    R- b = \frac{\beta}{1-\beta}\left[\int_{w\geq R}(w-R)d F(w)\right] \label{eq:reserv}
\end{equation}
The LHS is the cost of foregoing the current job. 
The RHS is the expected benefit of one more search.

Define the $g(R)$ as $$ g(R)\equiv \frac{\beta}{1-\beta}\left[\int_{w\geq R}(w-R)\mathrm{d}F(w)\right] $$
First order derivative of $g(R)$ is $$ g'(R) = -\frac{\beta}{1-\beta}(R-R)f(R)-\frac{\beta}{1-\beta}\left[\int_{w\geq R}dF(w)\right]= -\frac{\beta}{1-\beta}[1-F(R)]<0 $$ We have the unique solution to Eq~\eqref{eq:reserv}. 
Moreover, by the implicit function theorem, $$ \frac{dR}{db}= \frac{1}{1-g'(R)}>0 $$


% \paragraph{Wage Dispersion and Search}
% Start with eq~\eqref{eq:reserv}, $$ R-b = \frac{\beta}{1-\beta}\left[\int_{w\geq R}(w-R)\mathrm{d}F(w)\right] $$
% Rewrite this as 
% \begin{equation}
%     \begin{aligned}
%         R -b &= \frac{\beta}{1-\beta}\left[\int_{w\geq R}(w-R)\mathrm{d}F(w)\right]+\frac{\beta}{1-\beta}\left[\int_{w\leq R}(w-R)\mathrm{d}F(w)\right] \\ &-\frac{\beta}{1-\beta}\left[\int_{w\leq R}(w-R)\mathrm{d}F(w)\right],\\
%         &= \frac{\beta}{1-\beta}\left(\int_{\mathcal{W}}w dF(w)-R\right)-\frac{\beta}{1-\beta}\left[\int_{w\leq R}(w-R)d F(w)\right], 
%     \end{aligned} \label{eq:reserv2}
% \end{equation}











\end{document}