\documentclass[11pt,a4paper]{article}

\usepackage{graphicx,hyperref,amsmath,natbib,bm,url}
\usepackage[utf8]{inputenc} % include umlaute

% -------- Define new color ------------------------------
\usepackage{color}
\definecolor{mygrey}{rgb}{0.85,0.85,0.85} % make grey for the table
%   }}
\definecolor{darkblue}{rgb}{0.055,0.094,0.7}
\definecolor{darkred}{rgb}{0.6,0,0}

% --------------------------------------------------------
\usepackage{microtype,todonotes}
\usepackage[australian]{babel} % change date to to European format
\usepackage[a4paper,text={14.5cm, 25.2cm},centering]{geometry} % Change page size
\setlength{\parskip}{1.2ex} % show new paragraphs with a space between lines
\setlength{\parindent}{0em} % get rid of indentation for new paragraph
\clubpenalty = 10000 % prevent "orphans" 
\widowpenalty = 10000 % prevent "widows"

\hypersetup{pdfpagemode=UseNone} % un-comment this if you don't want to show bookmarks open when opening the pdf

\usepackage{mathpazo} % change font to something close to Palatino

\usepackage{longtable, booktabs, tabularx} % for nice tables
\usepackage{caption,fixltx2e}  % for nice tables
\usepackage[flushleft]{threeparttable}  % for nice tables

\newcommand*{\myalign}[2]{\multicolumn{1}{#1}{#2}} % define new command so that we can change alignment by hand in rows
 
% -------- For use in the bibliography -------------------
\usepackage{multicol}
\usepackage{etoolbox}
\usepackage{relsize}
\setlength{\columnsep}{1cm} % change column separation for multi-columns
% \patchcmd{\thebibliography}
%   {\list}
%   {\begin{multicols}{2}\smaller\list}
%   {}
%   {}
% \appto{\endthebibliography}{\end{multicols}}
% % --------------------------------------------------------


\usepackage[onehalfspacing]{setspace} % Line spacing

\usepackage[marginal]{footmisc} % footnotes not indented
\setlength{\footnotemargin}{0.1cm} % set margin for footnotes (so that the number doesn't stick out) 

\usepackage{pdflscape} % for landscape figures

\usepackage[capposition=top]{floatrow} % For notes below figure

% -------- Use the following two lines for making lines grey in tables -------------------
\usepackage{color, colortbl}
\usepackage{multirow}
% ----------------------------------------------------------------------------------------
\newtheorem{defn}{Definition}[section]
\newtheorem{reg}{Rule}[section]
\newtheorem{exer}{Exercise}[section]
\newtheorem{note}{Note}[section]
\newtheorem{expl}{Example}[section]
\newtheorem{pro}{properties}[section]
\newtheorem{asm}{Assumption}
\newtheorem{them}{Theorem}[section]
\newtheorem{rmk}{Remark}[section]
\newtheorem{coro}{Corollary}[section]
\newtheorem{lem}{Lemma}[section]
\newtheorem{pros}{Proposition}[section]


\usepackage{dsfont}
% \tcbset{highlight math style={enhanced,
%     colframe=red!60!black,colback=yellow!50!white,arc=4pt,boxrule=1pt,
\hypersetup{colorlinks=true,           % put a box around links
  linkbordercolor = {1 0 0}, % the box will be red
  pdfborder = {1 0 0},       % 
  % bookmarks=true,            % PDF will contain an index on the RHS
  urlcolor=darkred,
  citecolor=darkblue,
  linkcolor=darkred
}
% The following defines a footnote without a marker
\newcommand\blfootnote[1]{%
  \begingroup
  \renewcommand\thefootnote{}\footnote{#1}%
  \addtocounter{footnote}{-1}%
  \endgroup
}

% -------- Customize page numbering ----------------------------
\usepackage{fancyhdr}
\usepackage{lastpage}
 \pagestyle{fancy}
\fancyhf{} % get rid of header and footer
 \cfoot{ \footnotesize{ \thepage } }

% \rhead{\small{\textit{Jianqi Huang}}}
% \lhead{\small{\textit{Heterogeneous Agents'n OLG model}}}
\renewcommand{\headrulewidth}{0pt} % turn off header line
% --------------------------------------------------------------

\begin{document}
\section{Model}\label{sec:model}
\textbf{Motivation}: what determines the level of employment and unemployment in the economy. \\
\textbf{Textbook:} labor supply, labor demand, and unemployment as ``leisure'' .
Alternative: labor market frictions. \\ 
\textbf{Further question:} how should labor market frictions be modeled?
Search and matching: costly process of workers finding the right jobs.

\paragraph{McCall Partial Equilibrium Search Model}
\begin{itemize}
    \item The simplest model of search frictions.
    \item Problem of an individual getting draws from a given wage distribution.
    \item Decision: which jobs to accept and which to reject. 
    \item Sequential search typically more reasonable.
\end{itemize}
\paragraph{Environment}
Risk neutral individual, discrete time. Linear preferences: $$ \sum_{t=0}^\infty \beta^t c_t $$
Start as unemployed, with consumption equal to $b$. Wage offer $w$ drawn from a distribution $F(w)$. If she accepts, she will be employed at that job forever. Net Present Value of accepting a wage offer $w$ is $w/(1-\beta)$ since $$\frac{w}{1-\beta} = w + \beta w +\beta^2 w + \cdots.$$
We could use probability to describe decision rule $$ a_t : \mathcal{W}\to [0,1] $$ where $\mathcal{W}$ is the set of wage offers.
Dynamic programming problem: $$ V(w) = \max \left\{ \frac{w}{1-\beta}, b + \beta \int V(w') \mathrm{d}F(w') \right\} $$ 
(Indifference condition)
Reservation wage given by $$ \frac{R}{1-\beta} =  b+\int_{\mathcal{W}} V(w')\mathrm{d}F(w') $$
For all $w\geq R$, $V(w) = \frac{w}{1-\beta}$.
Therefore, $$ \int_{\mathcal{W}}V(w)\mathrm{d}F(w) = \frac{R F(R)}{1-\beta} +\int_{w\geq R}\frac{w}{1-\beta} \mathrm{d}F(w) $$
Combining with reservation wage equation, $$ \frac{R}{1-\beta} = b +\beta \left[\frac{R F(R)}{1-\beta} +\int_{w\geq R} \frac{w}{1-\beta}\mathrm{d}F(w)\right] $$
Rewriting, $$ \int_{w<R} \frac{R}{1-\beta} \mathrm{d}F(w) + \int_{w\geq R}\frac{R}{1-\beta}\mathrm{d}F(w) = b +\beta \left[\frac{R F(R)}{1-\beta} +\int_{w\geq R} \frac{w}{1-\beta}\mathrm{d}F(w)\right] $$
Subtracting $\beta R \int_{w\geq R} \mathrm{d}F(w)/(1-\beta)+\beta R \int_{w< R} \mathrm{d}F(w)/(1-\beta)$ from both sides,
$$ \begin{aligned}
    \int_{w<R} \frac{R}{1-\beta} \mathrm{d}F(w) + \int_{w\geq R}\frac{R}{1-\beta}\mathrm{d}F(w)-\beta R \int_{w\geq R} \mathrm{d}F(w)/(1-\beta)+\beta R \int_{w< R} \mathrm{d}F(w)/(1-\beta)\\ =  b+\beta\left[\int_{w\geq R}\frac{w-R}{1-\beta}d F(w)\right] 
\end{aligned}$$
Collecting terms, we obtain
\begin{equation}
    R- b = \frac{\beta}{1-\beta}\left[\int_{w\geq R}(w-R)d F(w)\right] \label{eq:reserv}
\end{equation}
The LHS is the cost of foregoing the current job. 
The RHS is the expected benefit of one more search.

Define the $g(R)$ as $$ g(R)\equiv \frac{\beta}{1-\beta}\left[\int_{w\geq R}(w-R)\mathrm{d}F(w)\right] $$
First order derivative of $g(R)$ is $$ g'(R) = -\frac{\beta}{1-\beta}(R-R)f(R)-\frac{\beta}{1-\beta}\left[\int_{w\geq R}dF(w)\right]= -\frac{\beta}{1-\beta}[1-F(R)]<0 $$ We have the unique solution to Eq~\eqref{eq:reserv}. 
Moreover, by the implicit function theorem, $$ \frac{dR}{db}= \frac{1}{1-g'(R)}>0 $$


% \paragraph{Wage Dispersion and Search}
% Start with eq~\eqref{eq:reserv}, $$ R-b = \frac{\beta}{1-\beta}\left[\int_{w\geq R}(w-R)\mathrm{d}F(w)\right] $$
% Rewrite this as 
% \begin{equation}
%     \begin{aligned}
%         R -b &= \frac{\beta}{1-\beta}\left[\int_{w\geq R}(w-R)\mathrm{d}F(w)\right]+\frac{\beta}{1-\beta}\left[\int_{w\leq R}(w-R)\mathrm{d}F(w)\right] \\ &-\frac{\beta}{1-\beta}\left[\int_{w\leq R}(w-R)\mathrm{d}F(w)\right],\\
%         &= \frac{\beta}{1-\beta}\left(\int_{\mathcal{W}}w dF(w)-R\right)-\frac{\beta}{1-\beta}\left[\int_{w\leq R}(w-R)d F(w)\right], 
%     \end{aligned} \label{eq:reserv2}
% \end{equation}

\subsection{Continuous time model with layoff}
The value function 
\begin{equation}
  U = b \Delta + e^{-r\Delta}[(1-e^{-f \Delta }) \int \max \{ V(w),U\} \mathrm{d}F(w) + e^{-f \Delta }U\label{eq:val}
\end{equation}
and 
\begin{equation}
  V(w) = \frac{w}{r} + e^{-r\Delta} [e^{s \Delta }V(w) + (1-e^{s \Delta })U] \label{eq:val2}
\end{equation}
Take the continuous time limit $\Delta \to 0$:
\begin{equation}
  r U = b + f \int\max \{V(w)- U,0\}\mathrm{d}F(w) \quad \text{and } r V(w) = w + s[V(w)-U] \label{eq:val3}
\end{equation}
Combining these two equations, we have
\begin{equation}
  r U = b + f \int \max \{\frac{w+sU}{r+s}-U,0\}\mathrm{d}F(w) \label{eq:val4}
\end{equation}
The reservation wage:
\begin{equation}
  \frac{R+ sU}{r+s} = U
\end{equation}
Combining the previous two equations to eliminate $U$, we have
\begin{equation}
  R - b = f \int_{w\geq R} \frac{w-R}{r+s} \mathrm{d}F(w)\label{eq:reserv}
\end{equation}
\textbf{LHS}: benefit of accepting a wage offer $R$. 
\textbf{RHS}: cost of accepting an offer $R$ = foregoing future better offer. 
At the optimum, two should be equated. 


\subsection{Mean-Preserving Spread}
Rewrite \begin{equation}
  R - b = f \int_{w\geq R} \frac{w-R}{r+s} \mathrm{d}F(w) - f \int_{w\leq R} \frac{w-R}{r+s}\mathrm{d}F(w) 
\end{equation}
Applying intergration by parts,
\begin{equation}
  \int_{w\leq R} w \mathrm{d}F(w) = R F(R) - \int_{w\geq R} F(w)\mathrm{d}w 
\end{equation}
Plugging back, 
\begin{equation}
  R - b = f \frac{w-R}{r+s} \mathrm{d}F(w) + f \frac{1}{r+s} \int_{w\leq R} F(w)\mathrm{d}w 
\end{equation}
We say distribution $\tilde{F}$ is a mean-preserving spread of $F$ \textbf{iff} $\mathbb{E}_{\tilde{F}}[w] = \mathbb{E}[w]$ and $\int^{\bar{w}} \tilde{F}(w)\mathrm{d}w > \int^{\bar{w}} F(w)\mathrm{d}w$ for all $\bar{w}$.  
It says that the mean is the same but the variance is higher. 

Reservation wage is now 
\begin{equation}
  \frac{r+s + f}{f}R - \frac{r+s}{f}b = \mathbb{E}[w] + \underbrace{\int^{R}F(w)\mathrm{d}w }_{\equiv h(R)} 
\end{equation}
note that
\begin{equation}
  h(0) = 0 , h'(R) = F(R)\in [0,1]
\end{equation}
when $F$ shift from $F$ to $\tilde{F}$, the reservation wage will increase. 
It is very similar to the American option, which says that accept the job offer only if the wage is high enough (``option value''). 

Therefore, you only care about the right tail of the wage distribution. \\ 
More variance/risk $\to$ more chances of a very good wage offer $\to $ search more. 

The rate at which workers transition from unemployed to employed is 
\begin{equation}
  UE = f(1-F(R))
\end{equation}
If $f$ increases,
\begin{equation}
  \frac{\mathrm{d}\ln UE }{\mathrm{d} \ln f} = 1- \frac{F'(R)R}{1-F(R)}\frac{\mathrm{d} \ln R }{\mathrm{d} \ln f}
\end{equation}
Under what condition $\frac{\mathrm{d}\ln UE }{\mathrm{d}\ln f } =0 $. 
consider that case: wage distribution follows Pareto distribution, $F(w) = 1-(w/\underline{w})^{-\alpha}$ and outside option $b$ is proportional to the average wage in the economy,
\begin{equation}
  b = \bar{b} E[w\mid w\geq R] = \frac{\bar{b}}{\alpha-1}R
\end{equation}
Plugging the condition into the equation~\eqref{eq:reserv}, we have
\begin{equation}
  R - b = \frac{f}{r+s} \frac{1}{\alpha-1} \underline{w}^\alpha (R)^{1-\alpha}
\end{equation}
Solving for $R$: 
\begin{equation}
  R = \left(\frac{f}{r+s}\frac{1}{\alpha-1}\underline{w}^\alpha\right)^{\frac{1}{\alpha}}
\end{equation}
The UE rate is 
\begin{equation}
  UE = (\alpha-1)(r+s)(1-\bar{b} \frac{\alpha}{\alpha-1})
\end{equation}

\paragraph{Diamond Paradox}
For all $\beta<1$ the unique equilibrium in the above economy is $R=0$, and all workers accept the first wage offer. 

\textit{Implication}: starting from an allocation where all firm offer $R$, any firm can deviate and offer a wage $R-\varepsilon$ and increase its profits. This proves that no wage $R>0$ can be an equilibrium. 

Solutions of Diamond Paradox:
\begin{itemize}
  \item Assume $F(w)$ is not the distribution of wages, but the distribution of ``fruits'' exogenously offered by ``trees''.
  \item Introduce other dimensions of heterogeneity
  \item Modify the wage determination assumptions
\end{itemize}

\paragraph{DMP model}
Matching function: $x(U,V)$ is CRS. 
$$ xL = x(uL,vL) \Rightarrow x= x(u,v) $$
Under this assumption, we can express everything as a function of the \textit{tightness} of the labor market. 
$$ q(\theta) = \frac{x}{v} = x\left(\frac{u}{v},1\right) $$ 
where $\theta$ is the tightness of the labor market. 
\begin{itemize}
  \item $q(\theta)$ is the job filling rate.
  \item $1/q(\theta)$ is the expected duration of unemployment.
  \item $\theta q(\theta)$ is the job finding rate for the unemployed.
  \item Job creation is $u \theta q(\theta )L$.
  \item Job destruction: exogenously given by $\delta (1-u)$.
\end{itemize}
$$ \delta(1-u) = u\theta q(\theta) $$
Therefore, $$ u = \frac{\delta}{\delta + \theta q(\theta)} $$
is called Beveridge curve. 
Equilibrium: unemployed flow equals to the job destruction flow.
First, we need to use Bellman equation of unemployed worker: 
$$ r U = b + p(\theta)E[W(w)-U] $$
Employed flow value: $$ r W(w) = w +\delta [U-W(w)] $$
We could solve these equations. However, we cannot pin down the wage function since we have 3 unknowns. 
Introduce the wage bargaining assumption: Nash bargaining. Before that, we need to consider the value function of firms. Assume each firm only has one position. Vacant flow value:
$$ r V = -\kappa + q(\theta)E[J(w)-V] $$
Matched flow value:
$$ r J(w) = (p-w) + \delta [V-J(w)] $$
Free entry equilibrium condition: $$ rV=0 \Rightarrow \frac{\kappa }{E[J(w)]} = q(\theta)$$
This is just a market clearing condition! 

Steady State: 
\begin{align}
  u &= \frac{s}{s+q\theta(q)}\\ 
  0 &= A f(k) - (r+\delta)k - w -\frac{(r+s)}{q(\theta)}\gamma_0 \\ 
  w &= (1-\beta )z + \beta [Af(k)-(r+\delta)k+\theta \gamma_0]\\
  Af'(k) &= r+\delta
\end{align}









\end{document}